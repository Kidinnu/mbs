% -*- program: xelatex -*-
%\documentclass[12pt, compress]{beamer} 
\documentclass[12pt, compress]{beamer}       
%\documentclass[notes=only]{beamer}

\usepackage{fontspec}
	
\setsansfont{PT Sans}
 \setmonofont{Liberation Mono}

\usepackage{beamerthemesplit}
\usefonttheme[onlymath]{serif}
%\DeclareFontEncoding{T1}
\fontencoding{T1}

%\usepackage[T2A]{fontenc} 
\usepackage[utf8x]{inputenc}
\usepackage[russian]{babel}
\usepackage{listings}

\usepackage{amsmath,amssymb,amsthm}
\hypersetup{unicode=true}
\usepackage{graphics,graphicx}
\usepackage{verbatim}
\usepackage{fancyvrb}
\usepackage{booktabs}
%\usepackage{media9}
\usepackage[super]{natbib}
\usepackage{color}
\definecolor{mygreen}{RGB}{28,172,0} % color values Red, Green, Blue
\definecolor{mylilas}{RGB}{170,55,241}
\definecolor{codecolor}{RGB}{0,120,0}
\definecolor{light-gray}{gray}{0.95}
\definecolor{light-green}{rgb}{0.93, 1, 0.8}
\definecolor{light-yellow}{rgb}{1,1,0.85}
\definecolor{dark-blue}{rgb}{0,0,0.6}
\definecolor{dark-red}{rgb}{0.7,0,0.0}
\definecolor{dark-green}{RGB}{10,150,0}

\usepackage{pifont}% http://ctan.org/pkg/pifont
\newcommand{\cmark}{\textcolor{green}{\ding{51}}}
\newcommand{\xmark}{\textcolor{red}{\ding{55}}}

\setbeamercolor{frametitle}{fg=dark-red,bg=gray!10}
\setbeamerfont{title}{series=\bfseries,parent=structure}
\setbeamerfont{frametitle}{series=\bfseries,parent=structure}

\setbeamertemplate{blocks}[rounded][shadow=false]
\setbeamertemplate{navigation symbols}{}

\newcommand{\code}[1]{\textcolor{dark-green}{\texttt{#1}}}
\newcommand{\cleartitlepage}[1]{\begin{frame}[plain]\begin{center}\textsc{#1}\end{center}\end{frame}}
\renewcommand{\emph}[1]{\textcolor{dark-blue}{#1}}
\newcommand{\emphb}[1]{\textcolor{dark-blue}{\textbf{#1}}}
\newcommand{\lstcomment}[1]{\textcolor{dark-green}{\# \texttt{#1}}}

\usepackage{dirtree}

\usetheme{CambridgeUS}
\usecolortheme{lily}

\lstset{basicstyle=\ttfamily,
    language=matlab,    
    keepspaces=true,
    extendedchars=\true,
    basicstyle={\small},
    breaklines=true,
    morekeywords={matlab2tikz},
    keywordstyle=\color{blue},
    morekeywords=[2]{1}, keywordstyle=[2]{\color{blue}},
    identifierstyle={ \bf \color{black} },
    stringstyle=\color{mylilas},
    commentstyle=\color{mygreen},%
    showstringspaces=false,
    numbers=left,%
    numberstyle={\scriptsize \color{gray}},
    numbersep=7pt, 
    emph=[1]{case,switch,otherwise,nonlocal,as,yield, with},emphstyle=[1]\color{blue}, 
    frame=single,    
    rulecolor=\color{red},    
    backgroundcolor=\color{light-gray},
    xleftmargin=0.3cm,
    frame=l,framesep=4pt,framerule=0.5pt,
    escapechar=|
    %lineskip=-1.0pt,
    %emph=[2]{word1,word2}, emphstyle=[2]{style},    
}

\usepackage{tikz}
\usetikzlibrary{positioning} 
\tikzset{cbutton/.style={rectangle,minimum width=10mm,thick,draw=red!50!black!50,
top color=white,bottom color=red!50!black!30}}
\tikzset{mbutton/.style={rectangle,minimum width=10mm,thick,draw=black!20,
top color=white,bottom color=black!30}}

\usetikzlibrary{arrows,shapes}
\tikzstyle{every picture}+=[remember picture]


\makeatother
\setbeamertemplate{footline}
{
  \leavevmode%
  \hbox{%  
  \begin{beamercolorbox}[wd=.3\paperwidth,ht=2.25ex,dp=1ex,center]{author in head/foot}%
    \usebeamerfont{author in head/foot}\insertshortauthor
  \end{beamercolorbox}%  
  \begin{beamercolorbox}[wd=.6\paperwidth,ht=2.25ex,dp=1ex,center]{title in head/foot}%
    \usebeamerfont{title in head/foot}\insertshorttitle
  \end{beamercolorbox}%
  \begin{beamercolorbox}[wd=.1\paperwidth,ht=2.25ex,dp=1ex,center]{date in head/foot}%
    \insertframenumber{} / \inserttotalframenumber\hspace*{1ex}
  \end{beamercolorbox}}%
  \vskip0pt%
}

\setbeamertemplate{headline}{}
\setbeamertemplate{footline}{}
 
\AtBeginSection[]{
  \begin{frame}[plain]
  %\vfill  
  \begin{tikzpicture}
  \useasboundingbox (0,0) rectangle(\the\paperwidth,\the\paperheight);  
  \fill[color=dark-red]   (-1cm, 3.9cm) rectangle(\the\paperwidth, 5.9cm);
  %\usebeamerfont{title}\insertsectionhead\par%
  %\node[text width=\the\paperwidth,align=center] at (current page.center) {\color{ExecusharesWhite}\Large\textbf{\insertsectionhead}};  
  \node[text width=\the\paperwidth,align=center] at (6cm,4.9cm) {\color{white}\Large\textbf{\insertsectionhead}};  
  \end{tikzpicture}
  \end{frame}
}



% ============================================================================

\title[numpy]{Модель орбитальной ступени}
\subtitle{Динамика твёрдого тела и систем тел}
\author[Самарский университет]{Юдинцев В. В.}
\institute{Кафедра теоретической механики\\Самарский университет}
\date{\today}

\usepackage{tikz}
\usetikzlibrary{positioning} 
\tikzset{cbutton/.style={rectangle,minimum width=10mm,thick,draw=red!50!black!50,
top color=white,bottom color=red!50!black!30}}
\tikzset{mbutton/.style={rectangle,minimum width=10mm,thick,draw=black!20,
top color=white,bottom color=black!30}}


\tikzstyle{block} = [rectangle, draw, fill=yellow!20, text width=5em, text centered, rounded corners, minimum height=5em]
\tikzstyle{subblock} = [rectangle, draw, fill=white, text width=2em, text centered, minimum height=1em]
\tikzstyle{module} = [rectangle, draw, fill=none, text width=5em, text centered, rounded corners, minimum height=5em]
\tikzstyle{line} = [draw, ->]

\begin{document}

{
\setbeamercolor{background canvas}{bg=light-yellow} 
\usebackgroundtemplate[background]{}
\begin{frame}[plain]
%\begin{figure}[h]
%  \centering
%  \includegraphics[width=0.3\textwidth]{../common/Python_logo_and_wordmark.pdf}
%\end{figure}
\maketitle
\end{frame}
}

% \frame{\frametitle{Содержание}\tableofcontents}

\begin{frame}[c, fragile]
\frametitle{Файл-функция \emph{Ax.m}}
Файл-функция матрицы поворота вокруг оси \emph{x}:
\lstinputlisting[language=matlab]{Ax.m}
\end{frame}

\begin{frame}[c, fragile]
\frametitle{Файл-функция \emph{Ay.m}}
Файл-функция матрицы поворота вокруг оси \emph{y}:
\lstinputlisting[language=matlab]{Ay.m}
\end{frame}

\begin{frame}[c, fragile]
\frametitle{Файл-функция \emph{Az.m}}
Файл-функция матрицы поворота вокруг оси \emph{z}:
\lstinputlisting[language=matlab]{Az.m}
\end{frame}

\begin{frame}[c, fragile]
\frametitle{Файл-функция \emph{get\_frustum.m}}
Файл-функция координат точек поверхности усечённого конуса:
\lstinputlisting[language=matlab]{get_frustum.m}
\end{frame}

\begin{frame}[c, fragile]
\frametitle{Файл-функция \emph{affine\_transform.m}}
Поворот и перемещение точек:
\lstinputlisting[language=matlab]{affine_transform.m}
\end{frame}

\begin{frame}[c, fragile]
\frametitle{Файл-функция \emph{draw\_upper\_stage.m}}
\begin{lstlisting}
function draw_upper_stage(bc, rc, A)

% Nozzle 
L1 = 1; R11 = 1; R12 = 0.3;
[x,y,z,c] = get_frustum(R11,R12,L1,[1,0,0]);
[x,y,z  ] = affine_transform(x-bc(1), ...
                             y-bc(2), ...
                             z-bc(3), rc, A);
surf(x,y,z,c);
\end{lstlisting}
\end{frame}

\begin{frame}[c, fragile]
\frametitle{Файл-функция \emph{draw\_upper\_stage.m}}
Двигательный отсек
\begin{lstlisting}
L2 = 1; R21 = 1.5; R22 = 2;

[x,y,z,c] = get_frustum(R21,R22,L2,[0.5,0.5,0.5]);

x = x + L1;

[x,y,z  ] = affine_transform(x-bc(1),...
                             y-bc(2),...
                             z-bc(3), rc, A);
surf(x,y,z,c);

fill3(x(1,:),y(1,:),z(1,:),c(1,:,:));
\end{lstlisting}
\end{frame}

\begin{frame}[c, fragile]
\frametitle{Файл-функция \emph{draw\_upper\_stage.m}}
Корпус
\begin{lstlisting}
L3 = 6; R3 = 2;

[x,y,z,c] = get_frustum(R3,R3,L3,[0.0,0.5,0.0]);

x = x + L1 + L2;

[x,y,z  ] = affine_transform(x-bc(1), ...
                             y-bc(2), ...
                             z-bc(3), rc, A);

surf(x,y,z,c);

fill3(x(2,:),y(2,:),z(2,:),c(2,:,:));

\end{lstlisting}
\end{frame}

\begin{frame}[c, fragile]
\frametitle{Файл-функция \emph{draw\_upper\_stage.m}}
Переходный отсек
\begin{lstlisting}
L4 = 1; R41 = 2; R42 = 2.5;

[x,y,z,c] = get_frustum(R41,R42,L4,[0.5,1.0,0.5]);

x = x + L1 + L2 + L3;

[x,y,z  ] = affine_transform(x-bc(1), ...
                             y-bc(2), ...
                             z-bc(3), rc, A);

surf(x,y,z,c);

\end{lstlisting}
\end{frame}

\begin{frame}[c, fragile]
\frametitle{Файл-функция \emph{draw\_upper\_stage.m}}
Оси координат
\begin{lstlisting}
xc = A*[0, 7;
        0, 0;
        0, 0] + repmat(rc,1,2);
yc = A*[0, 0;
        0, 5;
        0, 0] + repmat(rc,1,2);
zc = A*[0, 0;
        0, 0;
        0, 5] + repmat(rc,1,2);  

line(xc(1,:),xc(2,:),xc(3,:),'Color',[1,0,0]);
line(yc(1,:),yc(2,:),yc(3,:),'Color',[0,1,0]);
line(zc(1,:),zc(2,:),zc(3,:),'Color',[0,0,1]);

\end{lstlisting}
\end{frame}


\begin{frame}[c, fragile]
\frametitle{Файл-скрипт \emph{view\_rocket.m}}
Оси координат
\begin{lstlisting}
figure;
axis([-10 10 -10 10 -10 10]);
hold on;
box;
axis vis3d;
for i=1:360
    cla;
    draw_upper_stage([4;0;0], ...
                     [0;0;0], Ay(i*pi/180.0));
    shading flat;
    lighting gouraud;
    light('Position',[0 0 10]);
    getframe;
end
\end{lstlisting}
\end{frame}

\section{Кватернионы}

\begin{frame}[c, fragile]
\frametitle{Файл-скрипт \emph{quat.m}}
Определение координат кватерниона по заданному направлению и углу поворота
\begin{lstlisting}
function res = quat(e, phi)

res = [cos(phi*0.5), reshape(e,1,3)*sin(phi*0.5)];

\end{lstlisting}
$$
  \boldsymbol \Lambda = \cos \frac{\varphi}{2} + \boldsymbol{e} \sin \frac{\varphi}{2},
$$
Результат (\emph{res}) содержит 4 компоненты кватерниона.
\end{frame}


\begin{frame}[c, fragile]
\frametitle{Файл-скрипт \emph{quat\_conj.m}}
Вычисление сопряженного кватерниона
\begin{lstlisting}
function res = quat_conj(q)

res = ...

\end{lstlisting}
\begin{equation*}
\boldsymbol \Lambda = \lambda_0 + \boldsymbol \lambda, \ \overline{\boldsymbol \Lambda} = \lambda_0 - \pmb \lambda
\end{equation*}
\end{frame}


\begin{frame}[c, fragile]
\frametitle{Файл-скрипт \emph{quat\_mul.m}}
Умножение кватернионов
\begin{lstlisting}
function AB = quat_mul(A, B)

AB = [A(1)*B(1)-dot(...), A(1)*B(2:end)+...];

\end{lstlisting}
\begin{multline*} 
  \boldsymbol{\Lambda} \circ \boldsymbol{B} = (\lambda_0+{\pmb{\lambda}}) \circ (b_0+ {\boldsymbol{b}}) \\ 
= (\lambda_0 + \lambda_1 \boldsymbol i_1 + \lambda_2 \boldsymbol i_2 + \lambda_3 \boldsymbol i_3) \circ (b_0 + b_1 \boldsymbol i_1 + b_2 \boldsymbol i_2 + b_3 \boldsymbol i_3) = \\
\lambda_0 b_0 + \lambda_1 b_1 \boldsymbol i_1 \circ \boldsymbol i_1 + \lambda_2 b_2 \boldsymbol i_2 \circ \boldsymbol i_2 + \lambda_3 b_3 \boldsymbol i_3 \circ \boldsymbol i_3 + \\
+ \lambda_0 \boldsymbol b + b_0 \boldsymbol \lambda + \underbrace{\lambda_1 b_2 \boldsymbol i_3 - \lambda_1 b_3 \boldsymbol i_2 - \lambda_2 b_1 \boldsymbol i_3 + \lambda_2 b_3 \boldsymbol i_1 + \lambda_3 b_1 \boldsymbol i_2 - \lambda_3 b_2 \boldsymbol i_1}_{\pmb \lambda \times \boldsymbol b} = \\
= \underbrace{\lambda_0 b_0 - \pmb \lambda \cdot \boldsymbol b}_{\text{скалярная часть}} + \underbrace{\lambda_0 \boldsymbol b + \pmb \lambda b_0 + \pmb \lambda \times \boldsymbol b}_{\text{векторная часть}}.
\end{multline*}
Использовать функции \emph{cross} (векторное произведение) и \emph{dot} (скалярное произведение)
\end{frame}


\begin{frame}[c, fragile]
\frametitle{Файл-скрипт \emph{quat\_transform.m}}
Поворот вектора $\boldsymbol r$ при помощи кватерниона
\begin{lstlisting}
function rp = quat_transform(q, r)

rp = quat_mul(quat_mul(q,[0, r]),quat_conj(q));
rp = rp(2:end);

\end{lstlisting}
\begin{equation*}\label{eq:quatAd}
  \boxed{\boldsymbol R' = \boldsymbol \Lambda \circ \boldsymbol R \circ \overline{\boldsymbol \Lambda}} \quad |\boldsymbol \Lambda| = 1.
\end{equation*}
\end{frame}


\begin{frame}[c, fragile]
\frametitle{Файл-скрипт \emph{quat\_to\_mat.m}}
Матрица поворота, соответсвующая кватерниону  
\begin{lstlisting}
function a = quat_to_mat(q)

a = [ 2*(q(1)*q(1)+q(2)*q(2))-1, ..., ...;
      ..., ..., ...;
      ..., ..., ... ];

\end{lstlisting}
\begin{equation*}
  \boldsymbol A = 
\begin{bmatrix}
2(\lambda_0^2+\lambda_1^2)-1 & 2(\lambda_1 \lambda_2 - \lambda_0 \lambda_3)  & 2(\lambda_1 \lambda_3 + \lambda_0 \lambda_2) \\
2(\lambda_1 \lambda_2 + \lambda_0 \lambda_3) & 2(\lambda_0^2 + \lambda_2^2) - 1 & 2(\lambda_2 \lambda_3 - \lambda_0 \lambda_1) \\
2 (\lambda_1\lambda_3-\lambda_0\lambda_2) & 2(\lambda_2\lambda_3+\lambda_0\lambda_1) &   2(\lambda_0^2 + \lambda_3^2) - 1
\end{bmatrix}
\end{equation*} 
\end{frame}

\begin{frame}[c, fragile]
\frametitle{Файл-скрипт \emph{view\_rocket.m}}
Оси координат
\begin{lstlisting}
figure;
axis([-10 10 -10 10 -10 10]);
hold on;
box;
axis vis3d;
for i=1:360
 cla;
 q = quat([1 0 0], i*pi/180.0);
 draw_upper_stage([4;0;0],[0;0;0],quat_to_mat(q));
 shading flat;
 lighting gouraud;
 light('Position',[0 0 10]);
 getframe;
end
\end{lstlisting}
\end{frame}



\end{document}
